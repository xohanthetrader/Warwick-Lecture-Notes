\documentclass{article}
\usepackage{graphicx} % Required for inserting images
\usepackage{amsfonts}
\usepackage{amsthm}
\usepackage{amsmath}
\title{CS146}
\author{Haria}
\date{October 2025}

\begin{document}

\maketitle

\section{Lecture 1 - Basic Notation}
\newtheorem{example}{Example}
\newtheorem{definition}{Definition}
\subsection{Sums}
\subsubsection{Basic Sums}
Sigma notation makes use of the symbol $\displaystyle \sum$
Basic sigma notation makes use of of lower and upper bounds that a bound variable iterates through in order to represent the sum of multiple elements
\begin{example}
    \[\sum_{k=2}^{11}k! = 2! + 3! + 4! + 5! + \cdots + 10 ! + 11!\]
\end{example}
The lower bound is the initial condition and it ticks up by one to the upper bound and sums the expression with the values substituted in for the bound variable
\subsubsection{Sums over sets}
Instead of giving a range for the sum to sum over we can give it an arbitrary set of numbers to sum over
\begin{example}
    Let $A = \{1,2,\pi\}$
    \[\sum_{k \in A}2^{k^{2} - 1 } = 2^{1^{2} - 1} + 2^{2^{2} - 1} + 2^{\pi^{2} - 1}\]
\end{example}
If we let $E= \{2,4,6,8,\dots\}$ or be the set of even numbers we have two ways of expressing sums on this set
\[\sum_{k \in E}\frac{1}{2^k} = \sum_{k-even,k \ge 1}\frac{1}{2^k}\]
\subsubsection{Infinite Sums}
Some sums can have $\infty$ as their upper bound such as 
\[\sum_{k = 1}^{\infty}\frac{1}{2^k}\]
AS we know that the common ratio is in the radius of convergence, the infinite series is the limit of the partial sums. We know
\[\sum_{k = 1}^{n}x^k = \frac{1 - x^{n+1}}{1-x}\]
The infinite sum there for is the limit of this expression when $x$ is in the radius of convergence i.e $|x| < 1$
\[\lim_{n \to \infty}\sum_{k = 1}^{n}x^k = \lim_{n \to \infty}\frac{1 - x^{n+1}}{1-x} = \frac{1}{1-x}\]
\subsubsection{Empty Sums}
If we take a sum over the empty set it will always equal 0 (additive identity)
\[\sum_{k \in \emptyset}f(k) = 0\]
\subsection{Products}
Product notation works largely the same as summation notation and can be used for the expression of known functions such as the factorial
\[n! = \prod_{k = 1}^{n}k = 1 \times 2 \times \cdots \times n\]
We can also take products over a set. Let $X = \{1,2,3\}$
\[\prod_{k \in X}f(k) = f(1) \times f(2) \times (3)\]
We also have the fact a product over the empty set is 1 (multiplicative identity)
\[\prod_{k \in \emptyset}f(k) = 1\]
This also gives us the definition of $0! = 1$ as a sum from $1 \to 0$ is the same as a product over the empty set as there are no elements in that range
\subsection{Sets}
The first set we have is the natural numbers
\[\mathbb{N} = \{0,1,2,\dots\}\]
Note this definition includes zero which can be contentious. If we have $0 \in  \mathbb{N}$ then we can define the natural as such
\begin{itemize}
    \item $0 \in \mathbb{N}$
    \item $k \in \mathbb{N} \Rightarrow k + 1 \in \mathbb{N}$
\end{itemize}
This definition has a flaw as sets with more elements than just the natural numbers can fulfil this definition but it serves as a constructive definition.
\\
From the Natural Numbers (more so the integers which is just two copies of the naturals) we can define further sets
\[\mathbb{Q} = \left\{\frac{a}{b} : a \in \mathbb{Z},b \in \mathbb{Z},b \ne 0\right\}\]
And if we have the real numbers we can also define another set (defining the real numbers is the role of analysis)
\[\mathbb{C} = \left\{a + bi : a \in \mathbb{R},b \in \mathbb{R}\right\}\]
\section{Lecture 2 - Proofs}
\newtheorem{proposition}{Proposition}
Proofs are mathematical arguments that apply formal logical deductive steps to prove a proposition from a set of axioms.There are three general proof techniques being Direct proofs,Proof by Contradiction and Proof by Induction.
\subsection{DirectThe empty tuple is a value of () and it also has a type of ().
 Proof}
This is the simplest form of proof simply with direct application of arguments
\begin{proposition}
    For $n \ge 2$
    \[\sum_{k=2}^{n} = \frac{(n-1)(n-2)}{2}\]
\end{proposition}
\begin{proof}
    Fix $n \ge 2$ and let $S(n) = \sum_{k=2}^{n}$
    \[S(n) = 1 + 2 + \cdots + (n-1) + n\]
    \[S(n) = n + (n-1) + \cdots + 2 + 1\]
    \[2S(N) = (n+2) + (n + 2) + \cdots + (n+ 2) + (n + 2)\]
    \[2S(N) = (n-1)(n+2)\]
    \[S(n) = \frac{(n-1)(n+2)}{2}\]
\end{proof}
\subsection{Proof By Contradiction}
This techniques works by making the assumption of the negation of the proposition and from there deriving a contradiction
\begin{proposition}
    $\sqrt{2} \notin \mathbb{Q}$
\end{proposition}
\begin{proof}
    Assume that $\sqrt{2} \in \mathbb{Q}$ then there exists a coprime pair $x,y \in \mathbb{Z},y \ne 0$ such that $\sqrt{2} = \frac{x}{y}$
    \newline
    We can now say $x = 2y^{2}$
    \newline
    This suggest ${x^2}$ is even and there for $x$ is even therefore $x^2$ is a multiple of 4 meaning $y^2$ is even so $y $ is even so both $x,y$ share a factor of 2 and are as such not coprime yielding a contradiction so $\sqrt{2} \notin \mathbb{Q}$
\end{proof}
\section{Lecture 3 - Induction}
\subsection{Definitions}
An inductive proofs is a methods for proving things on ordered sets. This is most commonly $\mathbb{N}$ or $\mathbb{Z}^{+}$
\\
Let $C(n)$ be a predicate where $n \in \mathbb{N}$ then we know $\forall_{n \in \mathbb{N}}C(n)$ if we have 
\begin{enumerate}
    \item Base Case : $C(0)$
    \item Inductive Case : $C(k) \Rightarrow C(k+1)$
\end{enumerate}
\subsection{Example}
I have skipped the simple example and have instead skipped to the harder example.
\begin{proposition} \label{hard}
    Given $n \in \mathbb{Z}$ let $C(n) : n^3 - n \textit{ is divisible by three}$
\end{proposition}
Proposition \ref{hard} cant be done directly by induction as it is not well ordered (no least element for the base case). Instead we can split this into two problems. First we can prove is for $n \in \mathbb{N}$ then attempt to etend this proof to $\mathbb{Z}^-$.
\begin{proposition}
    Given $n \in \mathbb{N}$ let $C(n) : n^3 - n \textit{ is divisible by three}$
\end{proposition}
\begin{proof}
\phantom{HIII}
    \begin{itemize}
        \item Base Case : $C(0) : 0 \textit{ is divisible by three}$
        \item Inductive Case : (IH) $C(k) : k^3 - k \textit{ is divisible by three}$ , Aim : $C(k + 1) : (k + 1)^3 - (k + 1) \textit{ is divisible by three}$
        \begin{align*}
            (k+1)^3 - k &= k^3 + 3k^2 + 3k + k-1 \\
                        &= (k^3 - k) + 3(k^2 + k)
        \end{align*}
    \end{itemize}
    By the (IH) the first term is a multiple of three and adding it to another multiple of three gives overall a multiple of three.
\end{proof}
Now we want to extend this to negative integers
\begin{proof}
    We already know for $n \ge 0 : C(n) \textit{ is true}$, if $n \le 0$  then $n^3 - n = -((-n)^3 - (-n)$. The $(-n)$ terms are positive so the previous proof applies and the negative of a multiple of three is also a multiple of three.
\end{proof}
By combining both of these proofs we get a proof of Proposition \ref{hard}

\section{Lecture 4 - Recurrence}
\begin{definition}
    A recurrence relation is one where the sequence is defined in terms of previous terms. These will also generally have a base case.
\end{definition}
\begin{example}
    Consider the following sequence defined by series 
    \[a_n = \sum_{k=2}^{n}k = \frac{(n-1)(n+2)}{2}\]
    We can rewrite this as a recurrence relation as follows
    \[a_i = \begin{cases}
        2 ,&i=2 \\
        a_{i-1} ,&i>2
    \end{cases}\]
\end{example}
What we want is for a way to go from recurrence relations to closed forms. In order to do this lets consider the following recurrence
\[F_n = F_{n-1} + F_{n-2}\]
This will have boundary conditions $F_1 = 1$ and $F_2 = 1$. This can also be derived from a consideration of breeding pairs of rabbits. There is in general two methods that we can use to solve it, Characteristic polynomials and by generating functions (and a secret third way using matrices). Here we shall cover the first.
\subsection{Characteristic Polynomials}
\subsubsection{Fitting the recurrence}
WE start by assuming we have a solution of the form $\lambda^n$
\[\lambda^n = \lambda^{n-1} + \lambda^{n-2}\]
Assuming $\lambda \ne 0$ and rearranging we get
\[\lambda^2 - \lambda - 1 = 0\]
This yields two values for lambada of $\varphi$ and $\overline{\varphi}$ where $\varphi$ is the golden ratio. One thing to note is that neither of these though satisfy the boundary conditions.
\subsubsection{Boundary Conditions}
\begin{proposition}
    A linear combination of solutions to a recurrence relation is itself a solution.
\end{proposition}
\begin{proof}
    Here a proof for a second order relation is given but it is trivial to see how it generalises. Given a $\lambda_1$ and $\lambda_2$ such that 
    \[\lambda_1^n = s\lambda_1^{n-1} + r\lambda_1^{n-2}\]
    \[\lambda_2^n = s\lambda_2^{n-1} + r\lambda_2^{n-2}\]
    The terms
    \[\alpha\lambda_1^n + \beta\lambda_2^n\]
    Can be rewritten as 
    \[s(\alpha\lambda_1^{n-1} + \beta\lambda_2^{-1}) + r(\alpha\lambda_1^{n-2} + \beta\lambda_2^{n-2}\]
    By means of subbing in the fact they satisfy the relations. As such that can be seen to also satisfy the relation.
\end{proof}
\section{Lecture 5}
\subsection{Summary of the method}
The method is summarised as follows
\begin{itemize}
    \item Assume $\lambda^n$ is a solution
    \item Solve for lambda
    \item Take a general linear combination of these solutions
    \item  Solve against the boundary conditions
\end{itemize}
We can also express a generalised recurrence relation of order $k$ as follows
\[S_n = \sum_{r=1}^{k}a_rS_{k-r} + f(n)\]
This will come with $k$ boundary condition in order to be soluble
\subsection{Gaussian Elimination}
For higher order recurrence relations there will be n simultaneous equations to solve in order to fit to the boundary conditions. This can be slow so we would like an algorithmic method to be able to solve this. One such method is using Gaussian elimination. Gaussian elimination is concerned with the reduction of equations to triangular form. This is shown bellow.
\begin{align*}
    a_{1,n}x_n &= b_1 \\
    a_{2,n-1}x_{n - 1} + a_{2,n}x_n &= b_2 \\
    \cdots 
\end{align*}
The general idea is that the $k$th equation is expressed in terms of the last $k$ variables. The way this is done is by using the last equation to eliminate the first variable from each equation. Then the second last equation to eliminate the 2nd variable from the rest and so on. Once the entire process is done each variable is trivially solved for. \\
\textbf{WATCH A YOUTUBE VIDEO FROM THE ORGANIC CHEMISTRY TUTOR ON THIS}

\section{Lecture 6}
\subsection{Homogeneity}
As previously stated a general linear recurrence relation can be given as 
\[S_n = \sum_{r=1}^{k}a_rS_{k-r} + f(n)\]
If we have the fact that $f(n) \equiv 0$ we may say the equation is homogenous otherwise its non homogenous
\subsection{Complex roots}
Sometimes when solving for $\lambda$ we will find that it has complex values.
When this happens we have 3 solutions of what to do
\begin{enumerate}
    \item Just do it normally and solve for \[\alpha\lambda_1^n +\beta\lambda_2^n \]
    \item You can split alpha and beta into real and complex parts and solve for those separately \[(\alpha + \alpha'i)\lambda_1^n +(\beta +\beta'i)\lambda_2^n \]
    \item You can take advantage of the fact complex roots come in conjugate pairs and solve \[\alpha(\lambda_1^n + \lambda_2^n) + \beta(\lambda_1^n - \lambda_2^n)\] This has a closely related trigonometric form given as follows \[r^n(\alpha\cos n\theta + \beta \sin n \theta\]
    Where $r$ is the modulus of the root and $\theta$ is the argument. This is gotten by application of de Moivre's theorem.
\end{enumerate}
\end{document}
